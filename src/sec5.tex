%! TEX root = ../main.tex
\section{Results}\label{sec:results}
The analysis covers the range of data collected between December 2015 and March 2017, namely from \texttt{Run53} to \texttt{Run79}, excluding \texttt{Run66} and \texttt{Run68}, in the {\gerda} nomenclature. This corresponds to an exposure of 20.46 kg$\cdot$yr for the BEGes and 16.22 kg$\cdot$yr for the enriched coaxials. As mentioned in the previous chapters, data from natural coaxial detectors are not considered in this analysis, as the contribution to the $2\nbb$ spectrum is negligible. Due to the presence of the blinding window data corresponding to an energy deposition that lies within $\pm$25 keV around the $Q_{\beta\beta}$ are neglected in the Likelihood computation.

The fit range was optimized to include the largest subset of data possible. The lower limit was set just above the \ce{^{39}Ar} Q-value, corresponding to 570 keV, because of three factors: the scarce knowledge of the the contributions to the background spectra in that region, the relatively large uncertainty of the dead layer thickness on the coaxial detectors that strongly affects the low energy region and finally uncertainties related to the performance of the acquisition trigger energy threshold. The upper limit was set to 5.3 MeV to include the $\alpha$-region and better constraint the contribution of the $\alpha$-model in the $2\nbb$ region.

% {{{ P_VALUE COMPUTATION
\marginnote{p-value\\computation} According to the Eq.~\ref{eq:pvalue}, One should compute the integral of the complete posterior distribution in order to evaluate the p-value. However, despite the powerful numerical integration tools shipped with BAT, the calculation fails if the parameter space dimension is approximatively greater than 10, as in the case of such a rich set of background sources. Anyway, instead of trying to solve a hard numerical problem, one can adopt a different approach applying again the MCMC methods, as suggested in \cite{p-value}. With a dataset probability modeled as a product of Poisson terms, as the case of fitting an histogram, the highest probability occurs when each bin's content matches the theoretical prediction. We used the best-fit energy histogram to define the starting point for a Markov chain. Moving the bin contents randomly up or down and applying the usual Metropolis test a large number of experiments can be quickly simulated and the p-value extracted.

% }}}
% {{{ BINNING
\marginnote{binning} At the beginning, data were binned with a fixed-size 4 keV binning. Then, at a first glance to results, it was realised that there was a great discrepancy between the model and data around the visible $\gamma$-lines of \ce{^{40}K} and \ce{^{42}K} and then the p-value was very low. This is always the case when dealing with $\gamma$-lines split over two or three bins, where the reliability of the energy calibration and resolution curves is essential. Thus, a variable binning that takes into account this effect has been adopted. Bins containing the most visible $\gamma$-lines in data have been merged into one, in particular: the \ce{^{42}K} line at 1461 keV, the \ce{^{40}K} line at 1525 keV, the \ce{^{208}Tl} line at 2614 keV, and the \ce{^{214}Bi} lines at 609, 1764, 2204 and 2447 keV. A summary of the binning sizes is given in Tab.~\ref{tab:bin}.
\begin{table}
	\centering
	\caption{Bin sizes adopted in the considered energy range.}\label{tab:bin}
	\begin{tabular}{llc}
		\toprule
		Source	&	Line [keV]	&	bin size [keV]	\\
		\midrule
		\ce{^{42}K}	&	1461	&	12	\\
		\ce{^{40}K}	&	1525	&	12	\\
		\ce{^{208}Tl}	&	2614	&	12	\\
		\cmidrule{1-3}
		\multirow{4}{*}{\ce{^{214}Bi}}	&	609		&	12	\\
			&	1764	&	8	\\
			&	2204	&	8	\\
			&	2247	&	12	\\
		\cmidrule{1-3}
		else	&		&	4	\\
		\bottomrule
	\end{tabular}
\end{table}

% }}}
% {{{ PRIORS
\marginnote{priors} For the $2\nbb$ half-life $T_{1/2}^{2\nu}$ two prior distributions have been considered: a flat one between 1.74 and 2.09 $\cdot10^{21}$ yr, corresponding to $\propto1/y^2$ when fitting $y=1/T_{1/2}^{2\nu}$, and a gaussian one centered on the result of {\gerda} phase \textsc{i} \cite{gerda2nbb}:
\begin{equation}(1.926\pm0.095)\cdot10^{21}\;\text{yr}\label{eq:2nbbph1}\end{equation}
with a $\sigma$ equal to the quoted error. When fitting $y=1/T_{1/2}^{2\nu}$, this corresponds to a prior:
\[\propto \frac{1}{y^2}\exp\left[-\frac{\left(\frac{1}{y}-\mu\right)^2}{2\sigma^2}\right]\;.\]

Some of the results from the screening measurements of the materials, reported in Tab.~\ref{tab:screening}, have been used to set prior distributions on the background activities. A gaussian prior is set when a precise activity with error is specified, a flat distribution is set when only an upper limit, or no information, is available. A special treatment is applied for measurements concerning cable activities. In {\gerda}, four types of cables are used to transport signals and provide power supply, but only three of them were screened to study their radioactivity. Furthermore there are some specific cables that come from a separate batch that were not screened too. Thus, it has been decided not to use the available results to set priors but only to provide a post-analysis consistency check to the activities in the \textsc{MaGe} volume for the cables given by the fit. Activities from the three cable types reported in Tab.~\ref{tab:screening} were combined as follows: they were weighted considering their actual presence in terms of mass in {\gerda}. For the non-screened cable type, Tecnomec 2 mils, the same contamination of Tecnomec 3 mils was assumed.

\subsection*{Systematic uncertainties}
To give an estimate of the systematic error associated to the two-neutrino double-beta decay half-life and the upper limit on $\aof$ various sources of systematic uncertainties have to be considered. The following discussion refers to the model containing only the Standard Model double-decay mode, then a method to extend the conclusions to the Lorentz-violating case will be presented.
\begin{itemize}
	\item First of all, the choice of the binning size could lead to a systematic error. To estimate the impact of the choice on the stability of the result on $T_{1/2}^{2\nu}$ a meaningful upper limit for the bin size has been considered, for which the analysis was repeated. A lower limit was not considered because 4 keV is already the lowest bin size possible, considering the reliability of the energy calibrations and models of the energy resolution. The chosen upper limit to estimate this contribution is 20 keV, for which the analysis reported an increment of +2.1\% of $\Tnu$.
	\item Some contributions could arise from the reliability of the Monte Carlo simulations. The main sources can be two: the accuracy of the implementation of the geometry of {\gerda} in \textsc{MaGe}, e.g.~the rounding of the detector corners, and the accuracy of the particle propagation algorithms in \textsc{Geant4}, which depends on the uncertainties on cross-sections and final states. The two contributions were estimated in previous works to be 1\% \cite{gerda2nbb} and 2\% \cite{geant4sys1, geant4sys2, geant4sys3} respectively.
	\item The uncertainties on the active volume fraction $f_{av}$ can enter in the model in several ways. On one hand, they affect the simulations of decays inside the detectors, such as $2\nbb$, because the fraction of decays taking place in the active and in the dead part of the detector changes with $f_{av}$. The uncertainties on the values of $f_{av}$ for BEGe and enriched coaxial detectors are of the order $\sim$3\% and $\sim$6\%, as it can be deduced from Tab.~\ref{tab:gedet}. The analysis was repeated twice, by replacing the $2\nbb$ energy spectra with those generated with the lower and the upper limit for $f_{av}$. The results on $\Tnu$ are stable within $^{+6.0}_{-5.3}\%$.

		The uncertainty on the active volume fraction also plays a role for the shape of the energy spectrum due to \ce{^{42}K} decays on the n$^+$ surface. {\color{red}giustificare perchè non ne teniamo conto}
	\item The effect of the uncertainty on the enrichment fraction $F_{76}$ also affects the estimates of $T_{1/2}^{2\nu}$ and $\aof$. It was evaluated in a similar manner as the one coming from the $f_{av}$ uncertainty. The uncertainty on $f_{76}$ for the detectors is $\sim$1.5\% except for \texttt{ANG1}, \texttt{ANG2} and \texttt{ANG3}, for which it is around $\sim$3\%. Again, the analysis was reprocessed for lower and upper limits of $f_{76}$ and the result on $\Tnu$ was found stable within $^{+3.0}_{-2.9}$.
	\item A contribution could also arise from the reliability of data acquisition and selection, but it is expected to be very small. As possible sources, the calculation of the live time as well as reconstruction and trigger efficiencies have to be considered. Also, some unphysical events could still be present in the dataset. The impact of this component is expected to be not larger than 0.5\% for $T_{1/2}^{2\nu}$.
	\item A last contribution could come from the approximations used in \textsc{Decay0} to compute the $2\nbb$ energy spectrum. As mentioned before, this calculation is in principle more precise than the one with the Primakoff-Rosen approximation, and has been cross-checked in other experiments. Thus this contribution is expected to be negligible in the present analysis.
\end{itemize}

The single contributions to the systematic uncertainty are listed in Tab.~\ref{tab:sys}. The total systematic uncertainty on $T_{1/2}^{2\nu}$ was obtained by summing in quadrature the single contributions.

To compute the systematic uncertainty on the $\aof$ estimate the following method has been adopted. Once the total lower and upper systematic uncertainties on $T_{1/2}^{2\nu}$ are known, an asymmetric gaussian distribution representing the systematic uncertainty can be built. This distribution was composed by two normal distributions centered in $\mu=1$ with widths corresponding to the positive and negative part of the systematic uncertainty, respectively. In order to fold this function in the posterior probability distribution of $\aof$, for each entry $p$ in the posterior distribution, a random number $r$ following the systematic uncertainty distribution was generated. A new histogram was filled with the product $p\cdot r$. This new distribution represents the final posterior distribution, comprising the statistical and systematic uncertainties. The final result for $\aof$ was determined ad the 90\% quantile of this distribution (see Fig.~\ref{fig:aofpost}).
\begin{table}
	\centering
	\caption{Single contributions to the systematic uncertainty on $T_{1/2}^{2\nu}.$}\label{tab:sys}
	{\renewcommand{\arraystretch}{1.1}
	\begin{tabular}{lc}
			\toprule
			Contribution	&	Uncertainty [\%]	\\
			\midrule
			Binning			&	+2.1	\\
			MC geometry		&	$\pm$1	\\
			MC tracking		&	$\pm$2	\\
			Active volume fraction 	&	$^{+6.0}_{-5.3}$	\\
			Enrichment fraction		&	$^{+3.0}_{-2.9}$	\\
			Data acquisition and selection		&	$\pm0.5$	\\
			\midrule
			Total			&	$^{+7.4}_{-6.5}$	\\
			\bottomrule
	\end{tabular}}
\end{table}
\subsection*{Standard Model double-beta decay mode}\addcontentsline{toc}{subsection}{Standard Model double-beta decay mode}
In a first approach in studying the background model only the standard mode of $2\nbb$ was included, in order to compare the results on the half-life with the one found in literature. The presence of all the background sources suggested by the screening measurements, plus the \ce{^{207}Bi} which is known to be present by other studies, was tested in a first fit model. Then the parameters showing an exponential decay as a marginalised posterior, thus with only an upper limit, were gradually excluded from the fit to gain in the end a minimal background model. In this first analysis non-informative prior distributions are set for all parameters. The results of the fit with the minimal model and non-informative priors are given in Tab.~\ref{tab:res1}.

\begin{table}
	\caption{Results for the minimal background model with non-informative priors.}
	\centerline{%
	\begin{tabular}{lcccc}
		\toprule
		Source							&	Global mode	&	Marginalised mode	&	1$\sigma$ contour	&	BI [cts/(keV$\cdot$kg$\cdot$yr)]	\\
		\midrule
		$T_{1/2}^{2\nu}$ [$10^{21}$ yr]	&				&						&						&	--	\\
		\cmidrule{1-5}
		\textsc{fibers} [mBq/kg]		&				&						&						&		\\
		\ce{^{40}K}						&				&						&						&		\\
		\ce{^{212}Bi} + \ce{^{208}Tl}	&				&						&						&		\\
		\ce{^{214}Pb} + \ce{^{214}Bi}	&				&						&						&		\\
		\cmidrule{1-5}
		\textsc{holders} [mBq/kg]		&				&						&						&		\\
		\ce{^{40}K}						&				&						&						&		\\
		\ce{^{214}Pb} + \ce{^{214}Bi}	&				&						&						&		\\
		\ce{^{228}Ac}					&				&						&						&		\\
		\ce{^{60}Co}					&				&						&						&		\\
		\cmidrule{1-5}
		\textsc{cables} [mBq/kg]		&				&						&						&		\\
		\ce{^{40}K}						&				&						&						&		\\
		\ce{^{212}Bi} + \ce{^{208}Tl}	&				&						&						&		\\
		\ce{^{234\text{m}}Pa}			&				&						&						&		\\
		\cmidrule{1-5}
		\textsc{mini-shroud} [mBq/kg]	&				&						&						&		\\
		\ce{^{40}K}						&				&						&						&		\\
		\ce{^{207}Bi}					&				&						&						&		\\
		\cmidrule{1-5}
		\textsc{other} [mBq/kg]			&				&						&						&		\\
		\ce{^{42}K} in LAr				&				&						&						&		\\
		\ce{^{42}K} on \textsc{coax} p$^+$&				&						&						&		\\
		$\alpha$-model BEGe [cts]		&				&						&						&		\\
		$\alpha$-model \textsc{coax} [cts]&				&						&						&		\\
		\cmidrule{1-5}
		p-value							&	\multicolumn{4}{c}{0.953}	\\
		\bottomrule
	\end{tabular}
	}
	\label{tab:res1}
\end{table}
% {{{ COMMENTS
\marginnote{comments} The double-beta decay half-life is in great agreement with the one extracted from phase \textsc{i} data (\ref{eq:2nbbph1}). One of the first notable things about the results of the background model is that the \ce{^{40}K} activity is much greater than the expectations everywhere except for the holder mounting. In the case of fibers the measure from material screening is four orders of magnitude less than the output of the fit, for the cables this reduces approximatively to one order of magnitude while the result on the holder mounting is compatible with expectations. This is a clear indication for the presence of additional \ce{^{40}K} contaminations both in the vicinity of and at a higher distance from the detector array. As we do not have measures of the \ce{^{40}K} activity on the mini-shroud materials we cannot make statements on that, however, as all the {\gerda} instrumentation was built in low background conditions, such an high intrinsic contamination of the materials is not plausible. Instead, a later surface contamination, perhaps occurred during the assembling, is more realistic. For all the other activities the results are compatible with expectations, where available. A p-value of 0.95 strongly supports the reliability of the background model.

% }}}
% {{{ RESULTS WITH PRIORS
\marginnote{adding\\priors} The results on the \ce{^{40}K} activity on the fiber shroud and the cables clearly indicates the inadequacy of the screening measurements to describe the actual contamination. Hence, only the gaussian prior distribution on the \ce{^{40}K} activity on the holder mounting has been added to the fit, as the result on it perfectly fits the expectations. A gaussian prior on $T_{1/2}^{2\nu}$ has also been set. Results are given in Tab.~\ref{tab:res2}. The final result on $\Tnu$, with the statistic and systematic uncertainties, is
\begin{equation}\Tnu=(1.926\quad^{+}_{-\;\text{stat}}\quad^{+}_{-\;\text{sys}})\cdot10^{21}\;\text{yr}\;.\end{equation}

\begin{table}
	\caption{Results for the minimal background model with informative priors.}
	\centerline{%
	\begin{tabular}{lcccc}
		\toprule
		Source							&	Global	&	Marginalized	&	1$\sigma$	&	BI [cts/(keV$\cdot$kg$\cdot$yr)]	\\
		\midrule
		$T_{1/2}^{2\nu}$ [$10^{21}$ yr]	&				&						&						&	--	\\
		\cmidrule{1-5}
		\textsc{fibers} [mBq/kg]		&				&						&						&		\\
		\ce{^{40}K}						&				&						&						&		\\
		\ce{^{212}Bi} + \ce{^{208}Tl}	&				&						&						&		\\
		\ce{^{214}Pb} + \ce{^{214}Bi}	&				&						&						&		\\
		\cmidrule{1-5}
		\textsc{holders} [mBq/kg]		&				&						&						&		\\
		\ce{^{40}K}						&				&						&						&		\\
		\ce{^{214}Pb} + \ce{^{214}Bi}	&				&						&						&		\\
		\ce{^{228}Ac}					&				&						&						&		\\
		\ce{^{60}Co}					&				&						&						&		\\
		\cmidrule{1-5}
		\textsc{cables} [mBq/kg]		&				&						&						&		\\
		\ce{^{40}K}						&				&						&						&		\\
		\ce{^{212}Bi} + \ce{^{208}Tl}	&				&						&						&		\\
		\ce{^{234\text{m}}Pa}			&				&						&						&		\\
		\cmidrule{1-5}
		\textsc{mini-shroud} [mBq/kg]	&				&						&						&		\\
		\ce{^{40}K}						&				&						&						&		\\
		\ce{^{207}Bi}					&				&						&						&		\\
		\cmidrule{1-5}
		\textsc{other} [mBq/kg]			&				&						&						&		\\
		\ce{^{42}K} in LAr				&				&						&						&		\\
		\ce{^{42}K} on \textsc{coax} p$^+$&				&						&						&		\\
		$\alpha$-model BEGe [cts]		&				&						&						&		\\
		$\alpha$-model \textsc{coax} [cts]&				&						&						&		\\
		\cmidrule{1-5}
		p-value							&	\multicolumn{4}{c}{0.953}	\\
		\bottomrule
	\end{tabular}
	}
	\label{tab:res2}
\end{table}

% }}}
\subsection*{Lorentz-violating double-beta decay mode}\addcontentsline{toc}{subsection}{Lorentz-violating double-beta decay mode}
The minimal background model described in the previous section and the result on $\Tnu$ were used to evaluate the presence of the Lorentz-violating double-beta decay component simply adding the new spectrum shape with $\aof$ as the fitting parameter as described in \ref{sec:bayes}. The marginalized posterior distribution for $\aof$, in electron mass units, before and after folding in the systematic uncertainties distribution, is shown in Fig.~\ref{fig:aofpost}. The final result for $\aof$ was determined as the 90\% quantile of this distribution:
\begin{equation}\aof<3425 \text{GeV}\qquad(\text{90\% C.L.})\end{equation}
\begin{figure}
	\centering
	\includegraphics{img/aof.pdf}
	\caption{Posterior distribution of $\aof$ (in units of electron mass) before and after folding with the systematic uncertainty distribution.}\label{fig:aofpost}
\end{figure}

\marginnote{comments} {\color{red}Aspettiamo i risultati...}

\begin{table}
	\caption{Results for the minimal background model with informative priors and the Lorentz-violating component.}
	\centerline{%
	\begin{tabular}{lcccc}
		\toprule
		Source							&	Global	&	Marginalized	&	1$\sigma$	&	BI [cts/(keV$\cdot$kg$\cdot$yr)]	\\
		\midrule
		$T_{1/2}^{2\nu}$ [$10^{21}$ yr]	&				&						&						&	--	\\
		\cmidrule{1-5}
		\textsc{fibers} [mBq/kg]		&				&						&						&		\\
		\ce{^{40}K}						&				&						&						&		\\
		\ce{^{212}Bi} + \ce{^{208}Tl}	&				&						&						&		\\
		\ce{^{214}Pb} + \ce{^{214}Bi}	&				&						&						&		\\
		\cmidrule{1-5}
		\textsc{holders} [mBq/kg]		&				&						&						&		\\
		\ce{^{40}K}						&				&						&						&		\\
		\ce{^{214}Pb} + \ce{^{214}Bi}	&				&						&						&		\\
		\ce{^{228}Ac}					&				&						&						&		\\
		\ce{^{60}Co}					&				&						&						&		\\
		\cmidrule{1-5}
		\textsc{cables} [mBq/kg]		&				&						&						&		\\
		\ce{^{40}K}						&				&						&						&		\\
		\ce{^{212}Bi} + \ce{^{208}Tl}	&				&						&						&		\\
		\ce{^{234\text{m}}Pa}			&				&						&						&		\\
		\cmidrule{1-5}
		\textsc{mini-shroud} [mBq/kg]	&				&						&						&		\\
		\ce{^{40}K}						&				&						&						&		\\
		\ce{^{207}Bi}					&				&						&						&		\\
		\cmidrule{1-5}
		\textsc{other} [mBq/kg]			&				&						&						&		\\
		\ce{^{42}K} in LAr				&				&						&						&		\\
		\ce{^{42}K} on \textsc{coax} p$^+$&				&						&						&		\\
		$\alpha$-model BEGe [cts]		&				&						&						&		\\
		$\alpha$-model \textsc{coax} [cts]&				&						&						&		\\
		\cmidrule{1-5}
		p-value							&	\multicolumn{4}{c}{0.953}	\\
		\bottomrule
	\end{tabular}
	}
	\label{tab:resaof}
\end{table}
\clearpage
\subsection*{Acknowledgements}
