%! TEX root = ../main.tex
\section{Results}\label{sec:results}
The analysis covers the range of data collected between December 2015 and March 2017, namely from \texttt{Run53} to \texttt{Run79}, excluding \texttt{Run66} and \texttt{Run68}, in the {\gerda} nomenclature. This corresponds to an exposure of 20.46 kg$\cdot$yr for the BEGes and 16.22 kg$\cdot$yr for the enriched coaxials. As mentioned in the previous chapters, data from natural coaxial detectors are not considered in this analysis, as the contribution to the $2\nbb$ spectrum is negligible. Due to the presence of the blinding window data corresponding to an energy deposition that lies within $\pm$25 keV around the $Q_{\beta\beta}$ are neglected in the Likelihood computation.

The fit range was optimized to the present knowledge about the background model. The lower limit was set just above the \ce{^{39}Ar} Q-value, corresponding to 570 keV, because of the scarce knowledge of the contributions to the background spectra in that region. The upper limit was set to 5.3 MeV to include the $\alpha$-region and better constraint the contribution of the $\alpha$-model in the $2\nbb$ region.

% {{{ P_VALUE COMPUTATION
\marginnote{p-value\\computation} According to the Eq.~\ref{eq:pvalue}, One should compute the integral of the complete posterior distribution in order to evaluate the p-value. However, despite the powerful numerical integration tools shipped with BAT, the calculation fails if the parameter space approaches the 10-20 number of dimensions, as in the case of such a rich set of background sources. Anyway, instead of trying to solve a hard numerical problem, one can adopt a different approach applying again the MCMC methods, as suggested in \cite{p-value}. With a dataset probability modeled as a product of Poisson terms, as the case of fitting an histogram, the highest probability occurs when each bin's content matches the theoretical prediction. We used the best-fit energy histogram to define the starting point for a Markov chain. Moving the bin contents randomly up or down and applying the usual Metropolis test a large number of experiments can be quickly simulated and the p-value extracted.

% }}}
% {{{ BINNING
\marginnote{binning} At the beginning, data were binned with a fixed-size 4 keV binning. Then, at a first glance to results, it was realised that the model was very poorly performant around the visible $\gamma$-lines of \ce{^{40}K} and \ce{^{42}K}. This is always the case when dealing with $\gamma$-lines split over two or three bins, where the reliability of the energy calibration and resolution curves is essential. Thus, a variable binning that takes into account this effect has been adopted. Bins containing the most visible $\gamma$-lines in data have been merged into one, in particular: the \ce{^{42}K} line at 1461 keV, the \ce{^{40}K} line at 1525 keV, the \ce{^{208}Tl} line at 2614 keV, and the \ce{^{214}Bi} lines at 609, 1764, 2204 and 2447 keV.

% }}}
% {{{ PRIORS
\marginnote{priors} For the $2\nbb$ half-life $T_{1/2}^{2\nu}$ two prior distributions have been considered: a flat one between 1.74 and 2.09 $\cdot10^{21}$ yr, corresponding to
\[\propto\frac{1}{y^2}\]
when fitting $y=1/T_{1/2}^{2\nu}$, and a gaussian one centered on the result of {\gerda} phase \textsc{i} \cite{gerda2nbb}:
\[(1.926\pm0.095)\cdot10^{21}\;\text{yr}\]
with a $\sigma$ taken equal to the quoted error. When fitting $y=1/T_{1/2}^{2\nu}$, this corresponds to a prior:
\[\propto \frac{1}{y^2}\exp\left[-\frac{\left(\frac{1}{y}-\mu\right)^2}{2\sigma^2}\right]\;.\]

Some of the results from the screening measurements of the materials, reported in Tab.~\ref{tab:screening}, have been used to set prior distributions on the background activities. A gaussian prior is set when a precise activity with error is specified, a flat distribution is set when only an upper limit, or no information, is available. A special treatment is applied for measurements concerning cable activities. In {\gerda}, four types of cables are used to transport signals and provide power supply, but only three of them were screened to study their radioactivity. Furthermore there are some specific cables that come from a separate batch that were not screened too. Thus, it has been decided not to use the available results to set priors but only to provide a post-analysis consistency check to the activities in the \textsc{MaGe} volume for the cables given by the fit. Activities from the three cable types reported in Tab.~\ref{tab:screening} were combined as follows: they were weighted considering their actual presence in terms of mass in {\gerda}. For the non-screened cable type, Tecnomec 2 mils, the same contamination of Tecnomec 3 mils was assumed.

\subsection*{Standard Model double-beta decay mode}\addcontentsline{toc}{subsection}{Standard Model double-beta decay mode}
In a first approach in studying the background model only the standard mode of $2\nbb$ was included, in order to compare the results on the half-life with the one found in literature. The presence of all the background sources suggested by the screening measurements was tested in a first fit model, plus the \ce{^{207}Bi} which is known to be present by other studies. Then the parameters showing an exponential decay as a marginalised posterior, thus with only an upper limit, were gradually excluded from the fit to gain in the end a minimal background model. In this first analysis non-informative prior distributions are set for all parameters.

\begin{table}
	\caption{Results for the minimal background model with non-informative priors.}
	\centerline{%
	\begin{tabular}{lcccc}
		\toprule
		Source							&	Global mode	&	Marginalised mode	&	1$\sigma$ contour	&	BI [cts/(keV$\cdot$kg$\cdot$yr)]	\\
		\midrule
		$T_{1/2}^{2\nu}$ [$10^{21}$ yr]	&				&						&						&	--	\\
		\cmidrule{1-5}
		\textsc{fibers} [mBq/kg]		&				&						&						&		\\
		\ce{^{40}K}						&				&						&						&		\\
		\ce{^{212}Bi} + \ce{^{208}Tl}	&				&						&						&		\\
		\ce{^{214}Pb} + \ce{^{214}Bi}	&				&						&						&		\\
		\cmidrule{1-5}
		\textsc{holders} [mBq/kg]		&				&						&						&		\\
		\ce{^{40}K}						&				&						&						&		\\
		\ce{^{214}Pb} + \ce{^{214}Bi}	&				&						&						&		\\
		\ce{^{228}Ac}					&				&						&						&		\\
		\ce{^{60}Co}					&				&						&						&		\\
		\cmidrule{1-5}
		\textsc{cables} [mBq/kg]		&				&						&						&		\\
		\ce{^{40}K}						&				&						&						&		\\
		\ce{^{212}Bi} + \ce{^{208}Tl}	&				&						&						&		\\
		\ce{^{234\text{m}}Pa}			&				&						&						&		\\
		\cmidrule{1-5}
		\textsc{mini-shroud} [mBq/kg]	&				&						&						&		\\
		\ce{^{40}K}						&				&						&						&		\\
		\ce{^{207}Bi}					&				&						&						&		\\
		\cmidrule{1-5}
		\textsc{other} [mBq/kg]			&				&						&						&		\\
		\ce{^{42}K} in LAr				&				&						&						&		\\
		\ce{^{42}K} on \textsc{coax} p$^+$&				&						&						&		\\
		$\alpha$-model BEGe [cts]		&				&						&						&		\\
		$\alpha$-model \textsc{coax} [cts]&				&						&						&		\\
		\cmidrule{1-5}
		p-value							&	\multicolumn{4}{c}{0.953}	\\
		\bottomrule
	\end{tabular}
	}
	\label{tab:res1}
\end{table}
% {{{ COMMENTS
\marginnote{comments} The results of the fit with the minimal model and non-informative priors are given in Tab.~\ref{tab:res1}. The double-beta decay half-life is in great agreement with the one extracted from phase \textsc{i} data. One of the first notable things is that the \ce{^{40}K} activity is much greater than the expectations everywhere except for the holder mounting. In the case of fibers the measure from material screening is four orders of magnitude less than the output of the fit, for the cables this reduces approximatively to one order of magnitude while the result on the holder mounting is compatible with expectations. This is a clear indication for the presence of additional \ce{^{40}K} contaminations both in the vicinity of and at a higher distance from the detector array. As we do not have measures of the \ce{^{40}K} activity on the mini-shroud materials we cannot make statements on that, however, as all the {\gerda} instrumentation was built in low background conditions, such an high intrinsic contamination of the materials is not plausible. Instead, a later surface contamination, perhaps occurred during the assembling, is more realistic. For all the other activities the results are compatible with expectations, where available. The p-value strongly supports the reliability of the background model.

% }}}
% {{{ RESULTS WITH PRIORS
\marginnote{adding\\priors} The results on the \ce{^{40}K} activity on the fiber shroud and the cables clearly indicates the inadequacy of the screening measurements to describe the actual contamination. Hence, only the gaussian prior distribution on the \ce{^{40}K} activity on the holder mounting has been added to the fit, as the result on it perfectly fits the expectations. A gaussian prior on $T_{1/2}^{2\nu}$ has also been set. Results are given in Tab.~\ref{tab:res2}.

\begin{table}
	\caption{Results for the minimal background model with informative priors.}
	\centerline{%
	\begin{tabular}{lcccc}
		\toprule
		Source							&	Global	&	Marginalized	&	1$\sigma$	&	BI [cts/(keV$\cdot$kg$\cdot$yr)]	\\
		\midrule
		$T_{1/2}^{2\nu}$ [$10^{21}$ yr]	&				&						&						&	--	\\
		\cmidrule{1-5}
		\textsc{fibers} [mBq/kg]		&				&						&						&		\\
		\ce{^{40}K}						&				&						&						&		\\
		\ce{^{212}Bi} + \ce{^{208}Tl}	&				&						&						&		\\
		\ce{^{214}Pb} + \ce{^{214}Bi}	&				&						&						&		\\
		\cmidrule{1-5}
		\textsc{holders} [mBq/kg]		&				&						&						&		\\
		\ce{^{40}K}						&				&						&						&		\\
		\ce{^{214}Pb} + \ce{^{214}Bi}	&				&						&						&		\\
		\ce{^{228}Ac}					&				&						&						&		\\
		\ce{^{60}Co}					&				&						&						&		\\
		\cmidrule{1-5}
		\textsc{cables} [mBq/kg]		&				&						&						&		\\
		\ce{^{40}K}						&				&						&						&		\\
		\ce{^{212}Bi} + \ce{^{208}Tl}	&				&						&						&		\\
		\ce{^{234\text{m}}Pa}			&				&						&						&		\\
		\cmidrule{1-5}
		\textsc{mini-shroud} [mBq/kg]	&				&						&						&		\\
		\ce{^{40}K}						&				&						&						&		\\
		\ce{^{207}Bi}					&				&						&						&		\\
		\cmidrule{1-5}
		\textsc{other} [mBq/kg]			&				&						&						&		\\
		\ce{^{42}K} in LAr				&				&						&						&		\\
		\ce{^{42}K} on \textsc{coax} p$^+$&				&						&						&		\\
		$\alpha$-model BEGe [cts]		&				&						&						&		\\
		$\alpha$-model \textsc{coax} [cts]&				&						&						&		\\
		\cmidrule{1-5}
		p-value							&	\multicolumn{4}{c}{0.953}	\\
		\bottomrule
	\end{tabular}
	}
	\label{tab:res2}
\end{table}

% }}}
\subsection*{Lorentz violating double-beta decay mode}\addcontentsline{toc}{subsection}{Lorentz violating double-beta decay mode}
