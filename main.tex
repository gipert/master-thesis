%! TEX program = lualatex
%        File: main.tex
%     Created: Mar Apr 18 06:00 pm 2017 C
% Last Change: Mar Apr 18 06:00 pm 2017 C
%
\documentclass[12pt, a4paper]{book}
	%\usepackage{pdfsync}
	%\synctex1
	%\usepackage[binding=5mm]{layaureo}
	%\usepackage[showframe]{geometry}
	\pagestyle{plain}
	%\usepackage{syntonly}
	%\syntaxonly
	\usepackage[hidelinks]{hyperref}
	\usepackage{graphicx}
	\usepackage{subfigure}
	\usepackage{xcolor}
	\usepackage{amsmath, amssymb}
		\renewcommand{\epsilon}{\varepsilon}
		\renewcommand{\theta}{\vartheta}
		\renewcommand{\rho}{\varrho}
		\renewcommand{\phi}{\varphi}
	%%%%%%%%%%%%%%%%%%%%%%%%%%%%%%%%%%%%%%%%%
	\usepackage[UKenglish]{babel}
	\usepackage[utf8]{inputenc}
	\usepackage[T1]{fontenc}
	%\usepackage{lmodern}
	\usepackage[lining]{libertine}
	\usepackage[cmbraces, libertine, varg, smallerops]{newtxmath}
	\usepackage{bm}
	\usepackage{ebgaramond}
	\usepackage{microtype}
	\usepackage{titlesec}
		\titleformat*{\section}{\Large\scshape}
		\titleformat*{\subsection}{\large\scshape}
		\titleformat*{\subsubsection}{\scshape}
	%%%%%%%%%%%%%%%%%%%%%%%%%%%%%%%%%%%%%%%%%
	\usepackage{indentfirst}
	%\usepackage[parfill]{parskip}
	\usepackage{booktabs}
	\usepackage[labelfont=bf]{caption}
	\usepackage{multirow}
	\captionsetup[table]{position=top}
	\usepackage{tikz}
	\usepackage{pgfplots}
	\usepackage[mode=image]{standalone}
	%\usepackage{adjustbox}
	\newcommand{\gerda}{\textsc{Gerda}}
	\newcommand{\nbb}{\nu\beta\beta}
	\newcommand{\aof}{\mathring{a}_\text{of}^{(3)}}
	\usepackage[version=4]{mhchem}
	%\usepackage{listings}
	%\lstset{language=C++}
	%\lstset{
	%        basicstyle=\small\ttfamily,
	%        keywordstyle=\color{blue}\bfseries,
	%        commentstyle=\color{darkgray},
	%        stringstyle=\color{orange}
	%        }
	\usepackage{multicol}
\begin{document}
\begin{titlepage}
	\thispagestyle{empty}
	\begin{center}
	\includegraphics[width=3.5cm]{img/logo.pdf} \\
	\vspace{1cm}
	\textsl{Universit\`a degli Studi di Padova} \\
	\textsl{Dipartimento di Fisica e Astronomia ``Galileo Galilei''} \\
	\vspace{11pt}
	\textsc{Tesi di Laurea Magistrale} \\
	\vspace{3cm}
	\LARGE{Search for Lorentz and CPT symmetries violation in double-beta decay using data from the \textsc{Gerda} experiment}
	\end{center}
	%\vspace{1cm}
	%\begin{center}\textbf{Abstract}\end{center}
	%\small{In the last years a dedicated experimental program searching for neutrinoless double beta decay has started. A careful study of two-neutrino double beta decay is also performed by these experiments because it constitutes a background for the neutrinoless mode. The high precision of many experiments has motivated the formulation of different modes of double-beta decay so that experiments can also look for new physics through unconventional decay modes. In this work we evaluate the presence of a Lorentz and CPT violating double beta decay mode, governed by the $\aof$ parameter, in data coming from the \textsc{Gerda} experiment through a Bayesian statistical analysis.} \\
	%\small{A central goal in modern physics is the development of a unified theory of quantum mechanics and general relativity, many approaches have been developed to combine these two descriptions of nature. It was discovered that many formulations of quantum gravity foresee the breakdown of Lorentz and CPT (the combination of Charge, Parity and Time-reversal transformations) symmetries at the Planck scale, however, the Standard Model (SM) of particle physics assumes a complete invariance under Lorentz transformations and hence CPT transformations. Direct observations at the Planck scale are not yet possible, instead it is possible that physics beyond the SM at very high energies can produce effects at lower energies, observable in current experiments. Some theoretical models predict such low energy effects to show up in the neutrino sector: for example, such effects could produce some distortions in the energy spectrum of the two neutrino double-beta decay process. In this work we evaluate the presence of a Lorentz and CPT violating double-beta decay mode using data coming from the second Phase of the \textsc{Gerda} experiment at LNGS in Italy.} \\
	\vspace{3cm}
	\begin{multicols}{2}
	\noindent
	\textsl{Candidato} \\
	\textsc{Luigi Pertoldi}
	\columnbreak
	\flushright
	\textsl{Relatore} \\
	\textsc{Riccardo Brugnera} \\
	\vspace{5mm}
	\textsl{Corelatore} \\
	\textsc{Katharina von Sturm}
	\end{multicols}
\end{titlepage}
\tableofcontents
%\listoftables
%\listoffigures
%! TEX root = ../main.tex
\section{Introduction}
	The Standard Model of particle physics, which describes the particles we now believe to be fundamental and their interactions, represents one of the greatest successes of physics of the last century. From the beginning of the debate towards the particle or wave nature of light to the very recent discovery of the Higgs boson it allowed physicists to make powerful and precise prediction later strongly confirmed by experimental evidences. First came quantum electrodynamics, developed in the 1930s, then the model was unified with weak interactions by Glashow in 1961 and finally provided with the Higgs mechanism by Weinberg and Salam in 1967. The Standard Model also includes quantum chromodynamics, the theory of strong interactions. In the past few decades, however, experimental evidences brought to light new phenomena, not predicted by the Standard Model, that suggested the presence of some new physics beyond the well-established existing theoretical framework. Also, the usual prescriptions of the quantum field theory cannot produce a valid theory of quantum gravity, and nowadays the solution to this problem is an active research topic. The Standard Model is very far from being the last word: there are still many gaps in our understanding, as we shall see later.

	The initial purpose to unify special relativity and quantum mechanics has led to the establishment of one of the building blocks of the theory behind the Standard Model: the invariance of the theory with respect to linear transformations of space-time, called Lorentz transformations. Lorentz was one of the first to study the invariance laws of Maxwell's equations and some years later Einstein showed that the transformations were a natural consequence of the foundations of special relativity, namely the constantness of the speed of light in every inertial frame of reference. However, recent research activities in quantum gravity showed that this symmetry is in fact nothing sacred, and its breakdown at the Planck scale cannot be excluded. Lorentz symmetry is also strongly tied to CPT symmetry (the combination of Charge, Parity and Time-reversal transformations) by the CPT theorem. This theorem states that every local, relativistic quantum field theory must be CPT invariant.

	The neutrino is playing the role of a messenger of the new physics beyond the Standard Model. Studying the properties and interactions of neutrinos has been one of the most exciting and vigorous activities in particle physics and astrophysics ever since Pauli first proposed their existence in 1930. In spite of their weakly interacting nature, we have so far accumulated an enormous amount of knowledge about them. No experiments that have been performed so far have detected conclusive deviations from the Standard Model, except neutrino oscillation experiments, which have shown that neutrinos are massive and mixed. The understanding of how the neutrinos would gain tiny masses and how they are mixed is an extremely challenging task that we have to face. The consequences could make the SM an effective theory of the yet unknown theory beyond the Standard Model.

	An open question of fundamental importance concerns the nature of these particles, which could be either of Dirac or Majorana type (neutrino and anti-neutrino are distinct or the same particle). An attempt to address the problem is done by experiments looking for the neutrinoless mode of the double-beta decay. The double-beta decay, in its standard mode ($2\nbb$), consists in a nucleus that decays into a daughter nucleus with two electrons and two electron anti-neutrinos as a byproduct. If the neutrino is a Majorana particle then another mode may occur ($0\nbb$), in which neutrinos are not produced at all. Neutrinoless double-beta decay experiments are considered the most promising way to solve the enigma, although these events are controlled by very rare second-order weak interactions. One of these is the {\gerda} experiment, located at LNGS in Italy at a depth of 3500 m w.e. (water equivalent). Gerda submerses bare high-purity germanium detectors enriched in \ce{^{76}Ge} into liquid argon (LAr), which serves simultaneously as a shield against external radioactivity and as cooling medium, in order to substantially reduce background sources. In these types of experiments the source is equal to the detector which yields high detection efficiency, this allows to reach a superior energy resolution and enhance the ability to discriminate background from signal.

	A careful study of two-neutrino double-beta decay is also performed by these experiments because it constitutes a background for the neutrinoless mode. The high precision of many experiments has motivated the formulation of different modes of double-beta decay so that experiments can also look for new physics through unconventional decay modes. As quoted above, the spontaneous breakdown of the Lorentz symmetry is an interesting feature that can be accommodated by many candidate theories of quantum gravity, such as string theory. The general framework that incorporates operators that break Lorentz invariance in the SM is the Standard Model Extension (SME). This effective field theory parametrizes generic deviations from Lorentz invariance in the form of coordinate-invariant terms in the action by contracting operators of conventional fields with controlling coefficients for Lorentz violation. It should be noted that a subset of operators in the SME also break CPT symmetry. All quantum field operators for Lorentz violation involved in the propagation of neutrinos have been classified and enumerated. The effects of these operators show up in neutrino oscillations experiments and time-of-flight experiments. However, four operators, odd under CPT, cannot be detected in this way, instead, they must be accessed through physical processes that involve neutrino phase-space properties, such as quantum decays. The net effect on the energy spectrum of the decay products is a distortion regulated by a combination of the four operators' coefficients, denoted with $\aof$.

	In this work we study the summed energy spectrum of the electrons produced in $2\nbb$ detected by the {\gerda} experiment, in order to extract an upper limit for $\aof$. A bayesian approach will be used to fit the background model, developed in the last years by the {\gerda} collaboration, to data. In the first chapter all the theoretical notions that underlie the phenomenon under study will be detailed, then a description of the experimental setup will be given in chapter two. We provide a description of the analysis and the employed statistical tools in chapter three and finally the results in chapter four.

%! TEX root = /Users/luigipertoldi/Tesi/tex/main.tex
\chapter{Theory Review}
\section*{The two-neutrino double-beta decay}
The two-neutrino double-beta decay ($2\nbb$) processes, first suggested by M.~Goeppert-Mayer in 1935 \cite{PhysRev.48.512}, can be schematically defined as:
\begin{equation*}
	\begin{split}
		& \mathcal{N}(A,Z)\longrightarrow \mathcal{N}(A,Z+2)+2e^-+2\bar{\nu}_e \qquad [2\nu\beta^-\beta^-] \\
		& \mathcal{N}(A,Z)\longrightarrow \mathcal{N}(A,Z-2)+2e^++2\nu_e \qquad [2\nu\beta^+\beta^+]\\
	\end{split}
\end{equation*}
where $\mathcal{N}(A,Z)$ represents a nucleus with mass number $A$ and atomic number $Z$. A $2\nu\beta^-\beta^-$ ($2\nu\beta^+\beta^+$) process consists of the simultaneous $\beta^-$ ($\beta^+$) decay of two neutrons (protons) in the same nucleus. The processes are generated at second-order in the perturbative expansion of weak interactions in the Standard Model.
\begin{figure}
	\centering%
	\makebox[\textwidth]{%
		\includestandalone[width=0.5\textwidth]{img/2nbbfey}%
		\includestandalone[width=0.5\textwidth]{img/0nbbfey}%
	}%
	\caption{Feynman graphs for two-neutrino and neutinoless double-beta decay.}
	\label{fig:nbbfey}
\end{figure}

Since the $2\nbb$ decays have a four-body leptonic final state, the sum of the kinetic energies of the two decay electrons have a continuous spectrum from zero to the Q-value of the decay process (the recoil energy of the final nucleus is negligible), which is given by
\[Q_{\beta\beta}=M_i-M_f-2m_e\]
where $M_i$ and $M_f$ are, respectively, the masses of the initial and final nuclei (i.e. the energy levels of their ground states; if the transition occurs into an excited energy level of the final nucleus, $M_f$ must be replaced with the appropriate energy). 

A nucleus $\mathcal{N}(A,Z)$ can decay through a $2\nbb$ process if its ground state has an energy which is larger than the ground-state energy of the nucleus $\mathcal{N}(A,Z\pm2)$ plus twice the electron mass. Moreover, if a nucleus can decay through both the $\beta$ and $2\nbb$ processes, in practice the $2\nbb$ decay process is not observable, because its $\beta$ decay lifetime is much shorter than its $2\nbb$ decay lifetime (the half-life of $2\nbb$ is typically around $10^{19}-10^{24}$ yrs). Therefore, in practice the $2\nbb$ decay of a nucleus is observable only if its $\beta$ decay is energetically forbidden or strongly suppressed because of a large change of spin. The $\beta^-$ decay of a nucleus $\mathcal{N}(A,Z)$ is energetically forbidden if its ground-state energy is lower than the ground-state energy of the nucleus $\mathcal{N}(A,Z+1)$ plus the electron mass ($Q_{\beta^{-}}<0$). Typically, in $2\nu\beta^-\beta^-$ decays the energy levels of the three nuclei $\mathcal{N}(A,Z)$, $\mathcal{N}(A,Z+1)$, and $\mathcal{N}(A,Z+2)$ are of the type depicted in Fig.~\ref{fig:levelsGe76}, where the specific case of the \ce{^{76}Ge}, \ce{^{76}As}, and \ce{^{76}Se} nuclei is considered.
\begin{figure}
\centering
	\subfigure
		{\includestandalone[width=0.5\textwidth]{img/levelsGe76}}
	\subfigure
		{\includegraphics[width=5cm]{img/masspar}}
	\caption{Schematic illustration of the energy level structure of the $2\nu\beta^-\beta^-$ decay of \ce{^{76}Ge} into \ce{^{76}Se}.}
	\label{fig:levelsGe76}
\end{figure}

The naturally occurring isotopes which can decay through the $2\nu\beta^-\beta^-$ process, with forbidden or suppressed $\beta^-$ decay are 35, listed for example in \cite{Giunti:2007ry}. All of the initial and final nuclei in the $2\nu\beta^-\beta^-$ process are even-even, i.e. they have an even number of protons and neutrons. Their binding energy is larger than that of the intermediate odd-odd nuclei because of the pairing force acting between identical nucleons. For the same reason, all of the initial and final nuclei have a $0^+$ ground state. Therefore, all ground- state to ground-state transitions are $0^+\rightarrow0^+$. Ground-state transitions to an excited state of the final nucleus may be energetically allowed, as in the case of the $\ce{^{76}Ge}\rightarrow\ce{^{76}Se}$ $2\nu\beta^-\beta^-$ decay in Fig.~\ref{fig:levelsGe76}, in which there is an accessible $2^+$ excited state of \ce{^{76}Se}. However, due to a cancellation occurring in the phase space integral and the lower Q-value, the $0^+\rightarrow2^+$ double-beta decay is suppressed with respect to $0^+\rightarrow0^+$ \cite{Tomoda:1991}.

There are only six naturally occurring isotopes which can decay through the $2\nu\beta^+\beta^+$ process \cite{Haxton:1985am}. These isotopes have small Q-values and lifetimes which are much longer than the lifetimes of the $2\nu\beta^-\beta^-$. The reason for the rarity of $2\nu\beta^+\beta^+$-decaying isotopes and their small Q-values can be understood considering that that the decay $\mathcal{N}(A,Z)\rightarrow\mathcal{N}(A,Z-1)$ can occur either through the $\beta^+$ process $\mathcal{N}(A,Z)\rightarrow\mathcal{N}(A,Z-1)+e^++\nu_e$ or through the electron-capture process $e^-+\mathcal{N}(A,Z)\rightarrow\mathcal{N}(A,Z-1)+\nu_e$. Since $Q_{EC} = Q_{\beta^+}+2m_e$, the electron-capture process can occur even if the $\beta^+$ process is energetically forbidden ($Q_{\beta^+}<0$). Thus, in order have an energetically forbidden $\mathcal{N}(A,Z)\rightarrow\mathcal{N}(A,Z-1)$ transitions, the ground-state energy of $\mathcal{N}(A,Z)$ must be smaller than the ground-state energy of the nucleus $\mathcal{N}(A,Z-1)$ minus the electron mass ($Q_{EC}<0$). Considering as a reference energy the ground-state energy of the intermediate nucleus ($\mathcal{N}(A,Z+1)$ in $2\nu\beta^-\beta^-$ decays and $\mathcal{N}(A,Z-1)$ in $2\nu\beta^+\beta^+$ decays), the ground-state energy of the initial nucleus in a $2\nu\beta^+\beta^+$ decay must be at least $2m_e$ lower than in the case of a $2\nu\beta^-\beta^-$ decay. This implies that $2\nu\beta^+\beta^+$ decaying isotopes are more rare than $2\nu\beta^-\beta^-$-decaying isotopes. Moreover, for the same energy difference between the ground states of the intermediate and final nuclei, the energy difference between the ground states of the initial and final nucleus in a $2\nu\beta^+\beta^+$ decay is at least $2m_e$ lower than in the case of a $2\nu\beta^-\beta^-$ decay, leading to a correspondingly smaller Q-value. For these reasons, $2\nu\beta^+\beta^+$ decay has been less studied than $2\nu\beta^-\beta^-$ decay and in the following we will consider only $2\nu\beta^-\beta^-$ decays (we will simply refer to them with $2\nbb$). In any case, the neutrino properties of $2\nu\beta^+\beta^+$ decays are the same as those of $2\nu\beta^-\beta^-$ decays. Let us only mention that $\mathcal{N}(A,Z)\rightarrow\mathcal{N}(A,Z-2)$ transitions can occur not only through $2\nu\beta^+\beta^+$ processes, but also through the $EC\beta^+$ process $e^-+\mathcal{N}(A,Z)\rightarrow\mathcal{N}(A,Z-2)+e^++2\nu_e$ and the $2EC2\nu$ process $2e^-+\mathcal{N}(A,Z)\rightarrow\mathcal{N}(A,Z-2)+2\nu_e$.

The rate of $2\nbb$ can be calculated by invoking the recipe of the Fermi golden rule for simple $\beta$ decay. To a good approximation, the kinematic part (the phase space of the leptons emitted in the decay) and the nuclear part (the matrix element responsible for the transition probability between two nuclear states) can be factorized as
\[\Gamma^{2\nu}=G^{2\nu}(Q_{\beta\beta},Z)|\mathcal{M}^{2\nu}|^2\]
where $G^{2\nu}$ is obtained by integration over the phase space of four leptons emitted in the decay and can be calculated exactly. The nuclear matrix element $\mathcal{M}^{2\nu}$ deals with the nuclear structure of the transition and is much more difficult to evaluate. 

Denoting the 4-momentum of the two electrons and the two antineutrinos by $p^\alpha_i=(E_i,\mathbf{p}_i)$ and $q^\alpha_i=(\omega_i,\mathbf{q}_i)$, respectively ($i=1,2$), the relevant matrix element is given by
\[i\mathcal{M}=iG^2_FV^2_{ud}[\bar{u}(p_1)\gamma^\mu(1-\gamma_5)v(q_1)][\bar{u}(p_2)\gamma^\nu(1-\gamma_5)v(q_2)]J_{\mu\nu}-(p_1\leftrightarrow p_2)\]
The hadronic tensor $J_{\mu\nu}$ corresponds to the product of two nuclear currents written in the impulse approximation \cite{Tomoda:1991}. Including the implementation of the long-wave and closure approximation fot the hadronic tensor \cite{Tomoda:1991}, we obtain
\[\sum_\text{spin}|\mathcal{M}|^2=64G^4_F|V_{ud}|^4g^4_A(p_1\cdot p_2)(q_1\cdot q_2)|\mathcal{M}^{2\nu}|^2\;,\]
where the nuclear matrix element involves vector and axial couplings for Fermi and Gamow-Teller transitions in the form
\[g^2_A\mathcal{M}^{2\nu}=g^2_V\mathcal{M}^{2\nu}_F-g^2_A\mathcal{M}^{2\nu}_{GT}\;.\]

General methods for phase-space factor calculations in double-beta decay have been developed \cite{Doi:1981,Doi:1983,Tomoda:1991}. The phase-space factor is obtained by integration over all possible energies and angles of the leptons emitted in the decay. For the two-neutrino mode, these leptons are the two electrons and the two (anti)neutrinos:
\[G^{2\nu} \propto \int \text{d}^3p_1\text{d}^3p_2\text{d}^3q_1\text{d}^3q_2F(Z,E_1)F(Z,E_2)\delta(E_1+E_1+\omega_1+\omega_2-E_F-E_I)\;,\]
where $F(Z,E)$ is the Fermi function that describes the Coulomb effect on the outgoing electrons and $E_I$, $E_F$ are respectively the energies of the parent and the daughter nucleus.

In the Primakoff–Rosen approximation \cite{PrimakoffRosen} for the nonrelativistic Coulomb correction, the spectrum of the individual electrons can be analytically calculated:
\[\frac{d\Gamma}{dK}=\Lambda\cdot(K^5+10K^4+40K^3+60K^2+30K)(Q_{\beta\beta}-K)^5\;,\]
where $K$ is the sum of the kinetic energies of the two electrons in units of the electron mass. The overall constant factor is given by
\[\Lambda=\frac{G_F^4g_A^4|V_{ud}|^4F^2_\text{PR}(Z)m_e^{11}}{7200\pi^7}|\mathcal{M}^{2\nu}|^2\]
with $F_\text{PR}(Z)=2\pi\alpha/Z(\-e^{-2\pi\alpha Z})$.
\begin{figure}
	\centering
	\includestandalone{img/energyspectra}
	\caption{Energy spectra for different double-beta decay modes: in blue the two-neutrino mode, red the Lorentz violating mode, green the neutrinoless mode.}
	\label{fig:energyspectra}
\end{figure}

\section*{The neutrinoless double-beta decay}
	The neutrinoless double-beta decay processes ($0\nbb$) of the types
\begin{equation*}
	\begin{split}
		& \mathcal{N}(A,Z)\longrightarrow \mathcal{N}(A,Z+2)+2e^- \qquad [0\nu\beta^-\beta^-] \\
		& \mathcal{N}(A,Z)\longrightarrow \mathcal{N}(A,Z-2)+2e^+ \qquad [0\nu\beta^+\beta^+] \\
	\end{split}
\end{equation*}
which have been proposed by W.~H.~Furry in 1939 \cite{PhysRev.56.1184}, are forbidden in the minimal Standard Model, because the conservation of the total lepton number is violated by two units. Considering that today we know, from oscillations experiments, that neutrinos are instead massive particles, there are two ways to characterize them: they could be Dirac (as all the other fundamental particles) or Majorana particles. Being a Majorana particle, as first proposed by E.~Majorana \cite{Majorana1932}, means basically do not distinguish between particle and anti-particle. $0\nbb$ decays, in the standard interpretation, are possible if neutrinos are massive Majorana particles. In this case, a nucleus which can decay through a $2\nbb$ process can also decay through $0\nbb$ process, albeit with a different lifetime. Also the other double-beta decay modes mentioned above have their neutrinoless analog. However, as reviewed in \cite{Rodejohann:2011mu}, should be noted that there are many other well-motivated particle physics scenarios and frameworks that allow for $0\nbb$, treated as negligible contributions in the standard interpretation.

	Considerable experimental efforts are being dedicated to the detection of $0\nbb$, as such experiments represent the only practical way of establishing the nature of neutrino mass and therefore of shedding light on the mechanism of the tiny (but nonzero) neutrino mass generation established by neutrino oscillation experiments.


\bibliographystyle{unsrt}
\bibliography{bib-thesis}
\end{document}
