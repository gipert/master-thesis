%! TEX program = lualatex
\documentclass[11pt, a4paper]{article}
\pagestyle{plain}
\usepackage[UKenglish]{babel}
\usepackage[utf8]{inputenc}
\usepackage[T1]{fontenc}
\usepackage[lining]{ebgaramond}
\usepackage[libertine, varg]{newtxmath}
\usepackage{microtype}
\usepackage{amsmath, amssymb}
\usepackage{multicol}
\newcommand{\aof}{\mathring{a}_\text{of}^{(3)}}
\usepackage{scrextend}
\usepackage{mhchem}
%\usepackage{lineno}
%\linenumbers
\begin{document}
\pagenumbering{gobble}
\begin{center}
	\Large{\textsc{Search for Lorentz and CPT symmetries violation in double-beta decay using data from the \textsc{Gerda} experiment}} \\
	\vspace{1cm}
	\textsw{Luigi Pertoldi} \\
	\normalsize{\texttt{luigi.pertoldi@pd.infn.it}}
	\vspace{1cm}
\end{center}
\centerline{\large\textsc{abstract}}\vspace{8pt}
A central goal in Modern Physics is the development of a unified theory of Quantum Mechanics and General Relativity. Many approaches have been developed to combine these two descriptions of Nature. It was discovered that many formulations of Quantum Gravity foresee the breakdown of Lorentz and CPT (the combination of Charge, Parity and Time-reversal transformations) symmetries at the Planck scale. However, the Standard Model (SM) of Particle Physics assumes a complete invariance under Lorentz transformations and hence CPT transformations. Direct observations at the Planck scale are not possible yet, instead it is possible that Physics beyond the SM at very high energies could produce effects at lower energies, observable in current experiments. The general framework that incorporates operators that break Lorentz invariance in the SM is the Standard Model Extension (SME) [1, 2]. The effects of these operators generally show up in neutrino oscillations experiments and time-of-flight experiments. However, four operators, odd under CPT, cannot be detected in this way, instead, they must be accessed through physical processes that involve neutrino phase-space properties, such as quantum decays [3]. In this kind of experiments, concerning quantum decays with neutrinos, the net effect of these operators on the energy spectrum of the decay products is a distortion regulated by a combination of the four operators' coefficients, denoted with $\aof$.

The \textsc{Gerda} [4] experiment submerses bare high-purity germanium detectors enriched in \ce{^{76}Ge} into liquid argon (LAr), which serves simultaneously as a shield against external radioactivity and as cooling medium, in order to substantially reduce background sources. \textsc{Gerda} is currently running through its second phase, with some structural improvements and more active mass with respect to the first phase.
The aim of \textsc{Gerda} is to detect the neutrinoless mode of double-beta decay, in order to establish the Dirac or Majorana nature of the neutrino. A careful study of two-neutrino double-beta decay $2\nu\beta\beta$ is also performed by \textsc{Gerda}, in which the effects of a Lorentz-violating theory can be investigated [5]. In this work, we provide an estimate of the $2\nu\beta\beta$ half-life and of the $\aof$ parameter analysing data coming from the second phase of \textsc{Gerda}.

Background modeling is an essential step to extract and study double-beta decay data. Starting from the screening measurements of the radioactivity inside \textsc{Gerda}'s components, energy spectra of various background sources were simulated inside the apparatus taking into account its geometry. Contributions from \ce{^{238}U} and \ce{^{232}Th} decay chains as well as \ce{^{42}K} and \ce{^{40}K} were considered. Then the presence of the isotopes was tested by fitting the simulated spectra of different contributions to the measured energy spectrum with a Bayesian statistical analysis. Data from the screening measurements of the \textsc{Gerda} components were used to set prior distributions on the activities of the background sources, and a p-value was used to provide a goodness-of-fit criterion. The presence of all the hypothetical contaminations was tested in a maximal model that contains all possible contributions, then a minimal model was built ruling out step-by-step the sources indicated as possibly absent by the fitting procedure. The Background Index (BI), namely the number of counts over units of energy, mass and time in the Region of Interest (\textsc{RoI}) around the $Q_{\beta\beta}$ of $2\nu\beta\beta$, was estimated for all the background contributions.

As a first step, the half-life of $2\nu\beta\beta$ was extracted only considering the SM contribution, then the CPT-violating mode was included, and an upper limit on $\aof$ was computed. The results are in good agreement with the literature, in particular, the half-life of $2\nu\beta\beta$ is in excellent agreement with a previous estimate extracted from \textsc{Gerda} phase \textsc{i} data. The results on $\aof$ are compared with estimates from other experiments.
\vspace{2cm}
\begin{multicols}{2}
\noindent
\textsc{The Author}
\columnbreak
\flushright
\textsc{The Supervisor}
\end{multicols}
\vspace*{\fill}
\begin{labeling}{{[2]}}
\footnotesize
\itemsep-1ex
	\item[{[1]}] D. Colladay, V. A. Kosteleck\'y, \emph{CPT violation and the Standard Model}, 1997\\\texttt{10.1103/PhysRevD.55.6760}
	\item[{[2]}] D. Colladay, V. A. Kosteleck\'y, \emph{Lorentz-violating extension of the Standard Model}, 1998\\\texttt{10.1103/PhysRevD.58.116002}
	\item[{[3]}] V. A. Kosteleck\'y, J. D. Tasson, \emph{Prospects for Large Relativity Violations in Matter-Gravity Couplings}, 2009 \texttt{10.1103/PhysRevLett.102.010402}
	\item[{[4]}] K.~H.~Ackermann et al. \emph{The \textsc{Gerda} experiment for the search of $0\nu\beta\beta$ decay in $^{76}$Ge}, 2013\\\texttt{[10.1140/epjc/s10052-013-2330-0]}
	\item[{[5]}] Jorge S.~Diaz, \emph{Limits on Lorentz and CPT violation from double beta decay}, 2014\\\texttt{[10.1103/PhysRevD.89.036002]}
\end{labeling}
\end{document}
